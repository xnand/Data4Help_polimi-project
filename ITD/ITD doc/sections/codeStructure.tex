\documentclass[../main.tex]{subfiles}

\begin{document}

The root directory of the source code is "ITD". Inside of it there are three folders:

\begin{description}
	\item[ITD doc] Source code of this document.
	\item[TrackMeServer] Source code of the back-end server: Application Server Mobile Client, Application Server Data4Help, Database Server and Web Server components.
	\item[MobileApp] Source code of the mobile application and wearable module.
\end{description}

\subsection{TrackMeServer}

The four modules inside this directory all share the same basic structure: inside the component's directory (eg. ApplicationServerMobileClient) there is the main javascript file to be launched with node (eg. ApplicationServerMobileClient.js), the documentation source for the component if available (doc.yml), and a routes directory that futher defined funcionalities provided by the component.

A more directory named "common" contains two files used by the components:

\begin{description}
	\item[config.json] Contains configuration variables that can or must be set for the components to work, and can be overridden by environment variables (see chapter 6).
	\item[common.js] Contains functions often used by all other components; they are defined here to avoid code repetition and ease the development process.
\end{description}

One more directory names "support" contains files useful for development and evaluation. Three python scripts and a standalone node (javascript) file. The python scripts are written and tested for python 3 version 3.7.2 and require the "requests" package:

\begin{description}
	\item[cleanDatabase.py] Simply performs a GET to /dropALL on the DatabaseServer component to reset the database tables (see the DatabaseServer component description below).
	\item[populateDatabase.py] This script uses the ApplicationServerMobileClient and ApplicationServerData4Help components to register random model data in the system.
	\item[subscriptionHandler.py] A simple server to listen and write data to simulate a company subscription's funcionality.
	\item[AmulanceDispatcherSimulator/dispatcher.js] Node JS server that simulates the external ambulance dispatcher simulator, required for AutomatedSOS funcionality.
\end{description}

Every main component javascript file sets up the environment and binds the service to the address provided by the config file (or environment variable).

\subsubsection{DatabaseServer}

This component has one more file, "knex.js", that sets up the connection with the DBMS.

Inside the main file the tables of the database are created, if they don't exists, then the files in "routes" are sourced for the various funcionalities. Also, a temporary endpoint is defined (GET to /dropALL), only for development and evaluation purposes, that can be used to drop all current tables and recreate them empty.

The three files in "routes" directory contain the endpoints relative to their name, for example "dbs\textunderscore user.js" contains all the endpoints to handle user informations.

\subsubsection{ApplicationServerMobileClient \& ApplicationServerData4Help}

As the DatabaseServer component, the main file set up the environment and binds the service to the provided address; these two components also provide an endpoint (GET to /docs) to the documentation of their API. Then they source the .js file in "routes" folder that contain the implementation of their endpoints funcionality.

The doc.yml is the source code of the documentation, read by swagger when starting the component through the swagger-jsdoc node module.

\subsubsection{WebServer}

todo

\subsection{MobileApp}

todo

\subsubsection{TrackmeMobile}

todo

\subsubsection{TrackmeWear}

todo


\end{document}
