\documentclass[../main.tex]{subfiles}

\begin{document}

We can use "The World \& The Machine" model by M. Jackson and P. Zave to make an initial domain analysis. This allows us to understand which are the entities and the phenomena the machine cannot directly observe ("The World"), which are the ones the external world cannot directly see ("The Machine"), and the ones that are shared, i.e. the mean of interaction between the two.

\begin{figure}[h!]
	\includegraphics[width=\linewidth]{worldandmachine.jpg}
	\caption{Program analysis as per "The World \& The Machine" model}
	\label{fig:worldandmachine}
\end{figure}

\subsection{Product perspective}

\subsubsection{Class diagrams}

Following there are the class diagrams that give an overall description of our main three services through the use of a conceptual model. Only the relevant parts of the system are present in every one of them for the sake of clarity, but it is to be considered as a single model.
\newpage
\thispagestyle{empty} % yo hide page numbers
\begin{figure}[H]
	\includegraphics[width=\paperwidth, angle=90]{rasd_classdiag_d4h.jpg}
	\caption{Class diagram for the Data4Help service}
	\label{fig:classdiag_d4h}
\end{figure}
\newpage
\thispagestyle{empty}
\begin{figure}[H]
	\includegraphics[width=\paperwidth, angle=90]{rasd_classdiag_sos.jpg}
	\caption{Class diagram for the AutomatedSOS service}
	\label{fig:classdiag_sos}
\end{figure}
\newpage
\thispagestyle{empty}
\begin{figure}[H]
	\includegraphics[width=\paperwidth, angle=90]{rasd_classdiag_t4r.jpg}
	\caption{Class diagram for the Track4Run service}
	\label{fig:classdiag_t4r}
\end{figure}
\newpage

\subsubsection{State machine diagrams}

Through the use of the state machine UML model, we desribe the evolution of the states for the main components of the system.

\begin{figure}[H]
	\includegraphics[width=\linewidth]{rasd_statemachine_grpreq.jpg}
	\caption{Statechart of a request for a group of cutsomers data}
	\label{fig:statemachine_grpreq}
\end{figure}

\begin{figure}[H]
	\includegraphics[width=\linewidth]{rasd_statemachine_specreq.jpg}
	\caption{Statechart of a request for a single cutsomer's data}
	\label{fig:statemachine_specreq}
\end{figure}

\begin{figure}[H]
	\includegraphics[width=\linewidth]{rasd_statemachine_runevent.jpg}
	\caption{Statechart of a running event}
	\label{fig:statemachine_runevent}
\end{figure}

\subsection{Product functions}

Following are the requirements our system has to satisfy in order to provide the main functionalities.

\begin{minipage}{\textwidth}
\vspace{4mm}
{\bf Data4Help}
\begin{description}
	\item [G1]  Third parties can make requests for accessing single customer data.
	\begin{description}
		\item [R1] The system must forward any request from a company to single customer's data to the corresponding user.
	\end{description}

	\item [G2]  Third parties can make requests for accessing aggregate anonymous data specifiying filters.
	\begin{description}
		\item [R2] The system must reject any request for data regarding a group of customers when it cannot guarantee the anonimity of its components.
	\end{description}

	\item [G3]  Third parties can access stored data, for which a request has been approved.
	\begin{description}
		\item [R3] The system must periodically collect and store customer's data.
	\end{description}

	\item [G4]  Third parties can subscribe to new data, for which a request has been approved.
	\begin{description}
		\item [R4] The system must periodically update the accessible data with the newly collected one.
	\end{description}

	\item [G5]  Users can accept or refuse requests for their data from third parties.
	\begin{description}
		\item [R5] The system must notify the user of the request and permit him to accept or refuse it.
		\item [R6] The system must update the request status with the answer provided by the user.
	\end{description}

	\item [G6]  Third parties whose requests are accepted are not prevented from accessing data.
	\begin{description}
		\item [R7] The system must deny access to customer data for companies that don't have an approved request associated to it.
	\end{description}
\end{description}
\end{minipage}
\vspace{8mm}


\begin{minipage}{\textwidth}
{\bf AutomatedSOS}
\begin{description}
	\item [G7]  From the time a customer's parameters indicate a health emergency status, an ambulance is dispatched to his location in less than 5 seconds.
	\begin{description}
		\item [R8] The system must continuously check the data read from customers subscribed to AutomatedSOS.
		\item [R9] In case the data indicate an emergency for a customer, the system must dispatch the closest ambulance to his location.
	\end{description}

	\item [G8]  When an ambulance is dispatched, the customer is notified of its arrival.
	\begin{description}
		\item [R10] After an ambulance is dispatched, the system must notify the customer of its arrival.
	\end{description}
\end{description}
\end{minipage}
\vspace{8mm}


\begin{minipage}{\textwidth}
{\bf Track4Run}
\begin{description}
	% \item [G9]  Organizers can define the path for a run.
	% \begin{description}
	%
	% \end{description}

	\item [G10]  Runners can view and enroll in available runs.
	\begin{description}
		\item [R11] The system must show open events to the users.
		\item [R12] The system must allow users to sort and filter events by date and location.
	\end{description}

	\item [G11] Spectators can see the position of participants on the map during a run.
	\begin{description}
		\item [R13] The system must collect and provide in real time the position of participants on a map.
	\end{description}
\end{description}
\end{minipage}
\vspace{8mm}

\subsection{User characteristics}
\subsection{Assumptions, dependencies, constrains}

{\bf Domain assumptions}

\begin{description}

	\item [G1] Third parties can make requests for accessing single customer data.
	\begin{description}
		\item [D1] Third parties are able to identify specific customers by SSN or FC.
	\end{description}

	\item [G2] Third parties can make requests for accessing aggregate anonymous data specifiying filters.
	\begin{description}
		\item [D2] A group is considered anonymous if is composed by more than 1000 people.
	\end{description}

	\item [G3] Third parties can subscribe to new data, for which a request has been approved.
	\begin{description}
		\item %TODO how are requests approved???
	\end{description}

	\item [G4] Third parties can subscribe to new data, for which a request has been approved.
	\begin{description}
		\item
	\end{description}

	\item [G5]  Users can accept or refuse requests for their data from third parties.
	\begin{description}
		\item [D3]  A sent request is always received by the targeted user
	\end{description}

	\item [G6]  Third parties whose requests are accepted are not prevented from accessing data.
	\begin{description}
		\item
	\end{description}

	\item [G7]  From the time a customer's parameters indicate a health emergency status, an ambulance is dispatched to his location in less than 5 seconds.
	\begin{description}
		\item [D4] Health emergency status occurs when a AutomatedSOS customer parameters falls below or exceeds healthy boundaries.
		\item [D5] AutomatedSOS customers are always connected to GSM or internet.
		\item [D6] An ambulance is always available for dispatching and can reach the customer position.
	\end{description}

	\item [G8]  When an ambulance is dispatched, the customer is notified of its arrival.
	\begin{description}
		\item
	\end{description}

	\item [G9]  Organizers can create a new running event, prividing the required information.
	\begin{description}
		\item [D7] Organizers choose a valid path for the event.
		\item [D8] The event is legally autorized by authorities.
	\end{description}

	\item [G10]  Runners can view and enroll in available runs.
	\begin{description}
		\item []
	\end{description}

	\item [G11] Spectators can see the position of participants on the map during a run.
	\begin{description}
		\item
	\end{description}
\end{description}

\end{document}
