\documentclass[../main.tex]{subfiles}

\begin{document}
    Considering the RASD and DD documents, various tests were made, analyzing the goals and requirements defined in them, and also the ones in the ITD.

    \begin{description}
        \item [General behaviour] The application consistently crashes when rotating the device on some pages, such as "Your connections" on a business user screen.
        \item [Registration] When the user inserts valid data for all fields, the registration process goes fine, for both business and individual users. When one or more fields are invalid, the fields is marked with an error icon. However there seems to be no validation on the data inserted, as we successfully registered inserting a birth date set in the future, and an obviously wrong Identity Code.
        \item [Login] The functionality functions as described, logging the user and presenting the proper page in relation to the type of user (business or individual). However, there seems to be a bug in the android activites that happens when the following interactions are made: register as business user, logout, login as individual user. The application should present the main page for a logged in user, but it presents the register page instead. Going back and logging in again solves the issue. The same goes with the following interactions: log out of business user Data4Help session, then register a business user for AutomatedSOS service: the application should present the user with the page for AutomatedSOS service, but it displays the Data4Help service page. Closing and reopening the application solves the issue, as it automatically logins into the proper user and page.
        \item [Data simulation] Simulated data is successfully sent to the system, and is accessible to business users who made the proper request, and to whom the user permitted the access. Also the data is updated as soon as the user logs in and starts the simulation again.
        \item [Request sending] The requests for user data are successfully sent and received by the targeted individual user, who can then accept or refuse them. When accepted, they grant the business user who made the request access to such individual user data. Sending a request for an individual user that has already accepted a request for the same business user successfully notifies that such request has already been accepted, and does not make any further one as it is not necessary.
        When making a request for a non existent user, the request is not forwarded. Also, as planned, when an individual user refuses a request coming from the same business user twice, the business user is not authorized to make further requests to such individual user.
        \item [Global search] The feature works as expected, requiring that the search finds a minimum of two users to display data, and displaying an average of all individual user's data as specified in the other project documents. This threshold however is not defined in the ITD, but only in the source code.
        \item [AutomatedSOS] Slot generation works as intended. Users can subscribe to the service given the code generated by the company, then the company can login to the proper web page to view user's emergency data, while an error message is displayed if the inderted code is invalid. It is however not possible to test wether or not the AutomatedSOS service works, since it is unclear how to trigger an emergency.
    \end{description}

    All tests made inside the application pass, with the exception for "SeeConnectionsTest.parametersDisplayedTest" that fails in our environment.

    All requirements reported in the ITD are satisfied, except for requirement R32: we had no instructions on how to simulate an emergency to verify AutomatedSOS notification, and the code for this feature seems to be missing.

\end{document}
